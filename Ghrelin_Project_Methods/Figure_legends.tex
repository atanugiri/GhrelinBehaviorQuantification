\textbf{Figure 1.} Behavioral feature comparisons between saline and ghrelin conditions.\
\textbf{a.} Example trajectories from 5 saline-treated rats (left) and 5 rats treated with low-dose IBU (2x) (right) during the first 20 s of the task. Each color represents an individual animal. Compared to saline, 2x IBU rats exhibited more direct trajectories with reduced path curvature.
\textbf{b.} Representative velocity traces from first 10 seconds of the trial showing decreased movement in 2x IBU compared to saline.
\textbf{c.} Representative examples of head–body angle. Left: a saline-treated rat showing a lower head–body angle ($0$ rad). Right: a 2x IBU-treated rat exhibiting a higher head–body angle ($0.51$ rad). Angles were computed as the deviation between the head vector (teal) and the body axis (purple).
\textbf{d.} Trajectory curvature was significantly reduced in the 2x group relative to saline controls (T-test: $t=3.621$, $p=4.553e-04$), with a smaller but still significant reduction at high-dose IBU (10x) (T-test: $t=2.472$, $p=1.499e-02$).
\textbf{e.} Velocity also decreased significantly at 2x IBU (T-test: $t=2.891$, $p=4.703e-03$), but no difference was observed at 10x (T-test: $t=1.800$, $p=7.458e-02$).
\textbf{f.} In contrast, head–body angle was significantly increased at 2x IBU (T-test: $t=-1.993$, $p=4.851e-02$) but unchanged at 10X (T-test: $t=-0.575$, $p=5.663e-01$).
\textbf{g.} Saline vs 2× IBU and (\textbf{h}) Saline vs 10× IBU across combined and individual task conditions. Rows indicate tasks (combined or specific), and columns represent extracted features (curvature, velocity, head–body angle). Each cell shows the statistical outcome of saline vs IBU comparisons for that task and feature, with color denoting significance level (p value) with T-test and arrows indicating direction of change. Sample sizes are shown as $n_{\text{S,a}}$ (saline animals), $n_{\text{IBU,a}}$ (IBU animals), $n_{\text{S,t}}$ (saline trials), and $n_{\text{IBU,t}}$ (IBU trials).
For explanation of these behavioral patterns please see supplementary note.
For detailed statistical values and additional analyses, refer to the statistical document.

Bars show mean $\pm$; SEM, with individual session values overlaid as gray dots.

\textbf{Figure 2.} Behavioral feature comparisons between saline and ghrelin conditions in RECORD task.\
\textbf{a.} The approach rate did not differ significantly between saline and 2× IBU groups (T-test: $t=-0.219$, $p=0.827$).
\textbf{b.} Example trajectories from 5 saline-treated rats (left) and 5 rats treated with low-dose IBU (2x) (right) during the first 20 s of the RECORD trial. Each color represents an individual animal. Compared to saline, 2x IBU rats exhibited more direct trajectories with reduced path curvature.
\textbf{c.} Trajectory curvature was significantly reduced in the 2x IBU relative to saline controls (T-test: $t=4.600$, $p=4.402e-06$). Left: mean $\pm$ SEM; Right: box plots showing the distribution of individual session values.
\textbf{d.} Velocity also decreased significantly at 2x IBU (T-test: $t=3.531$, $p=4.189e-04$). Left: mean $\pm$ SEM; Right: box plots showing the distribution of individual session values.
\textbf{e.} IBU promoted ‘nest formation’ behavior, with trajectories clustering in restricted zones. In the RECORD task, a nest is defined as the rectangular strip consisting of the two highest reward levels. Example trajectories from saline (left) and IBU (right) groups are shown. A potential explanation is that animals were already familiar and overtrained with the task, reducing novelty and encouraging cost- or energy-saving strategies.
\textbf{f.} IBU animals spent more time in nest (left, T-test: $t=-6.073$, $p=1.381e-09$). Left: mean $\pm$ SEM; Right: box plots showing the distribution of individual session values.
\textbf{g.} IBU animals spent less time outside nest (left, T-test: $t=-6.073$, $p=1.381e-09$). Left: mean $\pm$ SEM; Right: box plots showing the distribution of individual session values.

\textbf{Figure 3.} Comparison of behavioral features across saline, inhibitory, and excitatory groups.\
\textbf{a.} For white animals, in simple tasks, the normalized number of detections in zone was not significantly different between saline and 2x IBU groups (1-way ANOVA, $F(1,82)=0.14$, $p=0.707$). Inhibition of the striosomes did not significantly alter the effects of IBU ($F_{\text{IBU vs Inh}}(1,85)=1.21$, $p_{\text{IBU vs Inh}}=0.274$; $F_{\text{Sal vs Inh}}(1,87)=2.87$, $p_{\text{Sal vs Inh}}=0.094$). Excitation of the striosomes also showed no significant differences relative to either IBU or saline ($F_{\text{IBU vs Exc}}(1,76)=0.03$, $p_{\text{IBU vs Exc}}=0.874$; $F_{\text{Sal vs Exc}}(1,73)=0.02$, $p_{\text{Sal vs Exc}}=0.880$). 
\textbf{b.} In complex tasks, 2x IBU significantly reduced detections compared to saline (1-way ANOVA, $F(1,50)=10.73$), $p=0.002$). Inhibition of the striosomes extinguish effect of IBU ($F_{\text{IBU vs Inh}}(1,46)=11.38$, $p_{\text{IBU vs Inh}}=0.001$; $F_{\text{Sal vs Inh}}(1,42)=0.26$, $p_{\text{Sal vs Inh}}=0.612$), excitation of striosome mimics effect of IBU ($F_{\text{IBU vs Exc}}(1,40)=0.24$, $p_{\text{IBU vs Exc}}=0.628$; $F_{\text{Sal vs Exc}}(1,32)=6.46$, $p_{\text{Sal vs Exc}}=0.016$).
\textbf{c.} Trajectory curvature was significantly reduced in the excitatory group relative to both saline and inhibitory groups (T-test: $t=4.662$, $p=1.009e-05$), while no difference was observed between saline and inhibitory (T-test: $t=0.360$, $p=7.199e-01$).
\textbf{d.} Velocity was also significantly lower in the excitatory group compared to saline and inhibitory (T-test: $t=4.088$, $p=8.940e-05$), with no difference between saline and inhibitory (T-test: $t=-1.909$, $p=5.890e-02$).
\textbf{e.} Head–body angle was significantly increased in the excitatory group compared to saline (T-test: $t=-2.195$, $p=3.121e-02$), but not different from inhibitory (T-test: $t=0.077$, $p=9.391e-01$).
\textbf{f.} Saline vs inhibitory (Inh) and saline vs excitatory (Exc) across combined and individual task conditions. Rows indicate tasks (combined or specific), and columns represent extracted features (curvature, velocity, head–body angle). Each cell shows the statistical outcome of saline vs Inh/Exc comparisons for that task and feature, with color denoting significance level ($p$ value) with T-test and arrows indicating direction of change. Sample sizes are shown as $n_{\text{S,a}}$ (saline animals), $n_{\text{Inh,a}}$ (inhibitory animals), $n_{\text{Exc,a}}$ (excitatory animals), and corresponding trial counts ($n_{\text{S,t}}$, $n_{\text{Inh,t}}$, $n_{\text{Exc,t}}$). For explanation of these behavioral patterns please see supplementary note.
For detailed statistical values and additional analyses, refer to the statistical document.

Bars represent mean $\pm$; SEM, with individual session values overlaid as gray dots.

\textbf{New fig (Black)} \textbf{c.} For black animals, in simple tasks (Food alone, Light alone, Toy alone), the normalized number of detections in zone was not significantly different between saline and ghrelin groups (1-way ANOVA, $F(1,69)=0.12$, $p=0.726$). \textbf{b.} In complex tasks (Food–light and Toy–light tradeoff), ghrelin significantly reduced detections compared to saline (1-way ANOVA, $F(1,54)=5.94$), $p=0.018$). Data are shown as mean $\pm$ SEM with individual animal values overlaid.